\documentclass{siamltex}   % Use [final] when ready.

\usepackage{siampaper}
\usepackage{braket}
\usepackage{enumerate} % added by TK

% For debugging.
%\usepackage{showframe}

%% The lineno packages adds line numbers. Start line numbering with
%% \begin{linenumbers}, end it with \end{linenumbers}. Or switch it on
%% for the whole article with \linenumbers after \end{frontmatter}.
%% \usepackage{lineno}

% Meta-information for the PDF file generated.
\pdfinfo{/Author (Michael P. Friedlander)
         /Title (Insert Title Here)
         /Keywords (keyword1, keyword2, keyword3)}

\title{
  Gauges and Conic Optimization
}
\markboth{M. P. FRIEDLANDER, I. MACEDO, and T. K. PONG}
         {GAUGES AND CONIC OPTIMIZATION}
\author{
  Michael P. Friedlander%
  \thanks{%
    Department of Computer Science,
    University of British Columbia,
    Vancouver, BC, Canada.
    E-mail: \mailto{mpf@cs.ubc.ca}.
    Research partially supported by an NSERC Discovery Grant.
  }
  \and
  Ives J. A. Mac\^edo%
  \thanks{%
    Department of Computer Science,
    University of British Columbia,
    Vancouver, BC, Canada.
    E-mail: \mailto{ijamj@cs.ubc.ca}.
    Research partially supported by an NSERC Discovery Grant.
  }
  \and
  Ting Kei Pong%
  \thanks{%
    Department of Computer Science,
    University of British Columbia,
    Vancouver, BC, Canada.
    E-mail: \mailto{tkpong@cs.ubc.ca}.
    Research partially supported by an NSERC Discovery Grant.
  }
}
\date{\today}

\begin{document}

  \maketitle

  \thispagestyle{plain}
  \pagestyle{myheadings}

  \begin{abstract}
    Insert abstract here.
  \end{abstract}
  \begin{keywords}
    gauges, duality, convex optimization, inverse problems
  \end{keywords}
  \begin{AMS}
    90C15, 90C25
  \end{AMS}

  \tableofcontents
  \listoftodos\relax


  \section{Introduction}

  Regularization is a common technique for introducing a desired
  structure into the solution of an optimization problem. For many
  modern applications in machine learning and signal processing, these
  regularization functions are often nonsmooth and highly
  structured. The aim of this paper is to describe a class of
  optimization problems that neatly captures a wide variety of
  regularization formulations.

  All of the problems that we consider can be expressed as
  \begin{equation}
    \label{eq:1}
    \minimize{x} \quad \kappa(x) \textt{subject to} x\in\Cscr,
  \end{equation}
  where $\Cscr\subseteq\R^{n}$ is a convex set, and
  $\kappa:\R^{n}\to\R\cup\{+\infty\}$ is a \emph{gauge} function,
  i.e., a nonnegative, positively homogeneous convex function). This
  class of problems admits a different kind of duality relationship
  based on the gauge structure of its objective.

  [It's at this point that we might want to simply introduce the
  primal-dual pair of gauge programs, and just mention that we'll
  discuss the relationship further.]

  \subsection{Examples}

  Some examples follow.
  \begin{example}[Norms and minimum-length solutions]
    Norms. Introduce unit ball.
  \end{example}

  \begin{example}[Atomic norms and sparse optimization]
    See Recht et al.  their ``atomic'' norm definition
    \[
    \norm{x}_{\Ascr} := \inf\{ \lambda\ge0 \mid x\in\lambda\conv(\Ascr)\}
    \]
    is a gauge, where $\Ascr$ is a set of ``atoms'' \cite{Chan:2012}.
  \end{example}

  \begin{example}[Submodular functions]
    Let $N=\Set{1,\ldots,n}$, and consider the set-function
    $f:2^{N}\to\R$, where $f(\emptyset)=0$. The function $f$ is
    submodular if
    \[
    f(A) + f(B) \ge f(A\cup B) + f(A\cap B)
    \]
    for all $A,B\subset V$. The \cite{Lovasz:1983}
    extension of $f$ can be expressed as
    \[
    \hat f(w) = \sup_{x\in P_{f}}\innerp{w}{x},
    \]
    where $B_{f}=\Set{x\in\R^{N}\mid x(N) = f(N), \text{\ and \
      }x(S)\le f(S),\ \forall S\subseteq N}$, with $x(S)=\sum_{i\in
      S}x_{i}$ is the base polyhedron of $f$. Note that $\hat f$ is
    the support function of the convex set $B_{f}$, and is therefore a
    gauge defined by $B^{\circ}_{f}$.
  \end{example}

  \begin{example}[Nonnegative conic programming]
    Let $\Kscr$ be a closed convex cone, and let $\Kscr^{*}$ denote
    its dual. Then the conic optimization problem
    \begin{equation}
      \label{eq:2}
      \minimize{x} \quad \innerp{c}{x} \quad\st\quad Ax = b,\ x\in\Kscr
    \end{equation}
    always has a nonnegative objective, and is a gauge optimization
    problem. This is a generalization of the nonnegative LP that
    discussed by Freund.
  \end{example}

  \section{Background}

  \subsection{Gauge functions}
  \label{sec:gauges}

  \begin{itemize}
  \item Alternative def's of gauge functions
  \item Conjugates, relationship to support functions,  subgradients
  \end{itemize}

  \subsection{Polarity operations}
  \label{sec:polarity-operations}

  \section{Polar and anti-polar calculus}
  \label{sec:polar-anti-polar}

  \begin{itemize}{\color{red}
  \item Background on polar calculus (eg, Rockafellar)}{\color{green}
  \item Derive the anti-polar versions}
  \end{itemize}

  In this section, we derive formulae for anti-polar sets of some simple sets.
  The following formula follows directly from the definition of polar functions.
  \begin{proposition}\label{sec3:prop2}
    Let $\Cscr = \{x:\; \rho(b - x)\le \sigma\}$ for some proper closed convex gauge function $\rho$, $0 < \sigma<\rho(b)$. Then
    \[
    \Cscr' = \{y:\; b^Ty - \sigma\rho^\circ(y)\ge 1\}.
    \]
  \end{proposition}
  \begin{proof}
    Notice that $y\in \Cscr'$ is the same as $y^Tx \ge 1$ for all $x\in \Cscr$. This is further equivalent to
    \begin{equation*}
      \begin{split}
        &\ y^T(x-b)\ge 1 - b^Ty,\ \ \ \mbox{for all} \ x \ {\rm s.t.}\ \rho(b - x)\le \sigma,\\
        \Leftrightarrow &\ y^T(b - x)\le b^Ty - 1,\ \ \ \mbox{for all} \ x \ {\rm s.t.}\ \rho(b-x)\le \sigma,\\
        \Leftrightarrow &\ \sigma\rho^\circ(y)\le b^Ty - 1,\\
      \end{split}
    \end{equation*}
    where the last equivalence follows from the definition of polar function. This completes the proof.
  \end{proof}

  Proposition~\ref{sec3:prop2} is very general since any closed convex set ${\cal D}$ containing the origin can be represented in the form of $\{x:\; \rho(x)\le 1\}$,
  with $\rho(x)=\inf\{\lambda> 0:\; \lambda^{-1}x\in {\cal D}\}$ being the Minkowski functional of ${\cal D}$. For example, we have the following simple corollary.
  \begin{corollary}\label{sec3:prop2immedcor}
    Let $\Cscr = \{x:\; x\in b + K\}$ for some closed convex cone $K$ and a vector $b\notin -K$. Then
    \[
    \Cscr' = \{y\in K^*:\; b^Ty \ge 1\}.
    \]
  \end{corollary}
  \begin{proof}
    Apply Proposition~\ref{sec3:prop2} with $\rho(x) = \delta_{-K}(x)$, the indicator function of the cone $-K$, and notice that
    in this case, $\rho^\circ(y) = \delta_{K^*}(y)$.
  \end{proof}

  However, in general, it may not always be easy to
  obtain an explicit formula for the Minkowski functional of a given closed convex set ${\cal D}$. Hence, we consider some antipolar calculus,
  which allows us to express the antipolar of a more complicated set in terms of the antipolars of its constituents.
  We start by looking at the antipolar of the image of $\Cscr$ under a linear map.
  \begin{proposition}\label{sec3:antipolarAC}
    Let $A$ be a linear map. Then
    \[
    (A\Cscr)' = (A^T)^{-1}\Cscr'.
    \]
    Furthermore, if ${\rm cl}(A\Cscr)$ does not contain the origin, then both sets above are nonempty.
  \end{proposition}
  \begin{proof}
    Note that $y\in (A\Cscr)'$ is equivalent to
    \begin{equation*}
      \begin{split}
        &\ y^T(Ac) \ge 1,\ \ \mbox{for all $c\in \Cscr$},\\
        \Leftrightarrow &\ (A^Ty)^Tc \ge 1,\ \ \mbox{for all $c\in \Cscr$}.
      \end{split}
    \end{equation*}
    The last relation is equivalent to $A^Ty \in \Cscr'$. Hence, we have $(A\Cscr)' = (A^T)^{-1}\Cscr'$.
    Furthermore, the assumption ${\rm cl}(A\Cscr)$ does not contain the origin implies that $(A\Cscr)'$ is nonempty. This completes the proof.
  \end{proof}

  As a corollary, we get the following result concerning the preimage of $\Cscr$.
  \begin{corollary}\label{sec3:cor1}
    Let $A$ be a linear map so that $A^{-1}\Cscr$ is nonempty. Then
    \[
    (A^{-1}\Cscr)' = {\rm cl}(A^T\Cscr'),
    \]
    and both sets are nonempty.
  \end{corollary}
  \begin{proof}
    First of all, it is clear that ${\rm cl}(A^T\Cscr')$ is nonempty. Moreover, since $\Cscr$ does not contain the origin, $A^{-1}\Cscr$
    is also a closed convex set that does not contain the origin. Hence, $(A^{-1}\Cscr)'$ is also nonempty.

    We next show that ${\rm cl}(A^T\Cscr')$ does not contain the origin. Suppose that $y\in A^T\Cscr'$ so that
    $y = A^Tu$ for some $u\in \Cscr'$. Then for any $x\in A^{-1}\Cscr$, we have
    $Ax\in \Cscr$ and thus
    \[
    y^Tx = u^TAx \ge 1,
    \]
    showing that $y\in (A^{-1}\Cscr)'$. Thus, we have $A^T\Cscr'\subseteq (A^{-1}\Cscr)'$ and consequently ${\rm cl}(A^T\Cscr')\subseteq (A^{-1}\Cscr)'$.
    Since the set $(A^{-1}\Cscr)'$ does not contain the origin, it follows that ${\rm cl}(A^T\Cscr')$ does not contain the origin.

    We now apply Proposition~\ref{sec3:antipolarAC} with $A^T$ in place of $A$ and $\Cscr'$ in place of $\Cscr$ to obtain that
    \begin{equation*}
      (A^T\Cscr')' = A^{-1}\Cscr''.
    \end{equation*}
    Taking antipolar on both sides of the above relation, we arrive at
    \begin{equation}\label{sec3:2sidepolar}
    (A^T\Cscr')'' = (A^{-1}\Cscr'')'.
    \end{equation}
    Since $\Cscr'$ is ray-like,
    it follows that ${\rm cl}(A^T\Cscr')$ is also ray-like. Since ${\rm cl}(A^T\Cscr')$ does not contain the origin, we conclude that
    $(A^T\Cscr')'' = {\rm cl}(A^T\Cscr')$. Moreover, recall that $\Cscr''$ is the set ${\rm cl}\{\lambda x:\; \lambda\ge 1, x\in C\}$;
    see, for example, \cite[Page~176]{McLinden:1978}. Hence,
    \[
    (A^{-1}\Cscr'')' = \left(\bigcup_{\lambda\ge 1}\lambda A^{-1}\Cscr\right)' = (A^{-1}\Cscr)',
    \]
    where the second equality can be verified directly from definition. The conclusion now follows from the above discussion and \eqref{sec3:2sidepolar}.
  \end{proof}

  We have the following further corollary.
  \begin{corollary}\label{sec3:cor2}
    Let $A$ be a linear map so that $A^{-1}\Cscr$ is nonempty. Suppose either
    \begin{enumerate}[{\rm (i)}]
      \item $\Cscr$ is polyhedral; or
      \item ${\rm ri}(\Cscr)\cap {\rm Range}(A)\neq\emptyset$;
    \end{enumerate}
    then $(A^{-1}\Cscr)'$ is nonempty and
    \begin{equation}\label{sec3:antipolarinvAC}
    (A^{-1}\Cscr)' = A^T\Cscr'.
    \end{equation}
  \end{corollary}
  \begin{proof}
    We will show that $A^T\Cscr'$ is closed under the assumption of this corollary.
    Then the conclusion follows immediately from Corollary~\ref{sec3:cor1}.

    Recall from Abrams theorem (see, for example, \cite[Lemma~3.1]{Berman:1973}) that $A^T\Cscr'$ is closed if
    and only if $\Cscr' + \ker(A^T)$ is closed. We will thus establish the closedness of the latter set.

    Suppose that $\Cscr$ is a polyhedral. Then it is routine to show that $\Cscr'$ is also a polyhedral and thus
    $\Cscr' + \ker(A^T)$ is closed. Hence, the conclusion of the corollary holds under this assumption.

    Finally, suppose that ${\rm ri}(\Cscr)\cap {\rm Range}(A)\neq\emptyset$.
    From \cite[Theorem~2.2.1]{AuT03} and the bipolar theorem, we have
    ${\rm cl}({\rm dom}(\sigma_{\Cscr'})) = ((\Cscr')_\infty)^*$, where $(\Cscr')_\infty$ is the recession cone of $\Cscr'$, which
    turns out to be just ${\rm cl}({\rm cone}(\Cscr'))$. From this, Lemma~\ref{sec4:lem1} and the bipolar theorem, we see further that
    \[
    {\rm cl}({\rm dom}(\sigma_{\Cscr'})) = [{\rm cl}({\rm cone}(\Cscr'))]^* = [\Cscr^*]^* = {\rm cl}({\rm cone}(\Cscr)),
    \]
    and hence ${\rm ri}({\rm dom}(\sigma_{\Cscr'})) = {\rm ri}({\rm cone}(\Cscr))$, thanks to \cite[Theorem~6.3]{Roc70}. Furthermore, the assumption that
    ${\rm ri}(\Cscr)\cap {\rm Range}(A)\neq \emptyset$ is equivalent to ${\rm ri}({\rm cone}(\Cscr))\cap {\rm Range}(A)\neq \emptyset$, since
    ${\rm ri}({\rm cone}(\Cscr)) = \bigcup_{\lambda > 0}\lambda{\rm ri}(\Cscr)$; see \cite[Page~50]{Roc70}. Thus, the assumption
    ${\rm ri}(\Cscr)\cap {\rm Range}(A)\neq\emptyset$ together with \cite[Theorem~23.8]{Roc70} imply that
    \begin{equation}\label{sec3:sum1}
      \Cscr' + \ker(A^T) = \partial \sigma_{\Cscr'}(0) + N_{{\rm Range}(A)}(0) = \partial(\sigma_{\Cscr'} + \delta_{{\rm Range}(A)})(0).
    \end{equation}
    In particular, $\Cscr' + \ker(A^T)$ is closed.
  \end{proof}

  We have the following immediate corollaries
  by combining Corollary~\ref{sec3:cor2} with Proposition~\ref{sec3:prop2} and Corollary~\ref{sec3:prop2immedcor}.
  {\color{red} Can further tweak the $A$...}
  \begin{corollary}
    Suppose that $\Cscr = \{x :\; \rho(b-Ax)\le \sigma\}$ for some proper closed convex gauge function $\rho$,
    a surjective linear map $A$ and $0 < \sigma < \rho(b)$.
    Then
    \[
    \Cscr' = \{A^Ty:\; b^Ty - \sigma\rho^\circ(y)\ge 1\}.
    \]
  \end{corollary}
  \begin{corollary}
    Suppose that $\Cscr = \{x:\; Ax - b\in K\}$ for some closed convex cone $K$, $-b\notin K$ and a surjective linear map $A$. Then
    \[
    \Cscr' = \{A^Ty\in K^*:\; b^Ty \ge 1\}.
    \]
  \end{corollary}

  \section{Gauge duality}
  \label{sec:gauge-duality}

  \subsection{Weak duality}
  \label{sec:weak-duality}

  \subsection{Strong duality}
  \label{sec:strong-duality}

  \begin{itemize}{\color{red}
  \item Freunds projection CQ: if this is ``new", it will be a new condition for Fenchel duality.}
  {\color{green}
  \item Version of strong duality sans ray-like property
    \begin{itemize}
    \item Show that the ray-like property is redundant
    \item ray-like needed only for bi-dual
    \item ray-like extension by bi-dual
    \end{itemize}}
  \end{itemize}

In this section, we re-derive the gauge duality theory developed in \cite{freund:1987} from the
Fenchel duality theory and relax the ray-likeness assumption on $\Cscr$.
Unless otherwise specified, the set $\Cscr$ is a nonempty closed convex set not containing the origin.
Our main tool is the following so-called antigauge function
  \begin{equation}\label{defg}
    g(y) := \sup\left\{\alpha \ge 0:\; y = \alpha w \mbox{ for some }w\in \Cscr', \alpha \ge 0\right\},
  \end{equation}
  which was also used in  \cite{freund:1987} to relate the gauge duality theory to the duality theory of \cite{Gwinner:1985}.
  The following lemma gives some simple properties of this function $g$, where we provide proofs for completeness.
  \begin{lemma}\label{sec4:lem0}
    The following properties hold for the function $g$:
    \begin{enumerate}[{\rm (i)}]
      \item ${\rm dom}(g) = {\rm cone}(\Cscr') = \bigcup_{\lambda\ge 0}\lambda \Cscr'\neq \emptyset$, on which $g$ is finite;
      \item The supremum in the definition of $g(y)$ is attained for any $y\in {\rm dom}(g)$;
      \item $y\in \Cscr'$ if and only if $g(y)\ge 1$;
      \item $g$ is positively homogeneous;
      \item $y\in \Cscr'$ implies that $\frac{1}{g(y)}y\in \Cscr'$.
    \end{enumerate}
  \end{lemma}
  \begin{proof}
    It is clear that if $y\notin {\rm cone}(\Cscr')$, then $g(y) = -\infty$. On the other hand, if $y\in {\rm cone}(\Cscr')$, then
    for any fixed $c\in \Cscr$, we have $c^Ty \ge \alpha c^Tw\ge \alpha$ whenever $y = \alpha w$ for some $w\in \Cscr'$ and $\alpha\ge 0$.
    This implies that $0\le g(y)\le c^Ty < \infty$, i.e., $y\in {\rm dom}(g)$ and $g(y)$ is finite.

    To see that the supremum is attained, first notice that for any nonzero $y\in {\rm dom}(g)$, there exists $\underline{\alpha}> 0$
    and $w\in \Cscr'$ so that $y = \underline{\alpha}w$. This shows that one can restrict attention to any maximizing sequence with $\inf\{\alpha_t\}\ge\underline{\alpha}$.
    Then, along this maximizing sequence, there exists $w^t\in \Cscr'$ with $w^t = y/\alpha_t$. Since $\{\alpha_t\}$ is uniformly bounded
    away from zero, the sequence $\{w^t\}$ is bounded and hence one
    can obtain a convergent subsequence. The attainment of the supremum now follows from a standard argument. On the other hand, if $y=0$, the
    supremum is clearly attained.

    To prove {\rm (iii)}, we note first that $y\in \Cscr'$ clearly implies that $g(y)\ge 1$. On the other hand, if $g(y)\ge 1$,
    then $y = g(y)w = w + (g(y)-1)w$ for some $w\in \Cscr'$. Since $\Cscr'$ is ray-like, we conclude that $y\in \Cscr'$.

    Positive homogeneity follows directly from definition. Finally, if $y\in \Cscr'$, then $g(y)\ge 1> 0$ and so
    $g(\frac{y}{g(y)})=1$ by positive homogeneity. The conclusion of {\rm (v)} now follows from {\rm (iii)}.
  \end{proof}

  The next lemma gives a relation between the positive polar of $\Cscr$ and its antipolar. {\color{red} Move this lemma to background.}

  \begin{lemma}\label{sec4:lem1}
    It holds that $\Cscr^* = {\rm cl}({\rm cone}(\Cscr'))$, where $\Cscr^*$ is the positive polar of $\Cscr$, i.e.,
    $\Cscr^* = \{y:\; x^Ty \ge 0,\ \ \forall x\in \Cscr\}$.
  \end{lemma}
  \begin{proof}
    It is clear that ${\rm cl}({\rm cone}(\Cscr'))\subseteq \Cscr^*$. To show the converse inclusion, take any $x\in \Cscr^*$ and fix an $x_0\in \Cscr'$.
    Then for any $t > 0$, we have
    \begin{equation*}
      c^T(x + tx_0) \ge tc^Tx_0\ge t
    \end{equation*}
    for all $c\in \Cscr$, showing that $x + tx_0\in {\rm cone}(\Cscr')$. Taking limit as $t$ goes to $0$ shows that $x\in {\rm cl}({\rm cone}(\Cscr'))$,
    which completes the proof.
  \end{proof}

  The next lemma gives an alternative representation of the function $g$. The case when $\Cscr$
  is ray-like was discussed in \cite[Corollary~14B]{McLinden:1978}.

  \begin{lemma}\label{sec4:lem2}
    For any $y$, we have $\inf_{c\in \Cscr}c^Ty\ge g(y)$. Moreover,
    \begin{equation}\label{sec4:eq}
      \inf_{c\in \Cscr}c^Ty = \begin{cases}
        g(y)& {\rm if\ }y\in {\rm dom}(g),\\
        0& {\rm if\ }y\in {\rm cl}({\rm dom}(g))\backslash ({\rm dom}(g)).
      \end{cases}
    \end{equation}
    If, in addition, $\Cscr$ is ray-like, then $\inf_{c\in \Cscr}c^Ty= g(y)$
    for all $y\notin {\rm cl}({\rm dom}(g))\backslash ({\rm dom}(g))$.
  \end{lemma}
  \begin{proof}
    For any $y\notin {\rm dom}(g)$, the inequality $\inf_{c\in \Cscr}c^Ty\ge g(y)$ holds trivially.
    On the other hand, if $y\in {\rm dom}(g) = {\rm cone}(\Cscr')$, then
    for any fixed $c\in \Cscr$, we have $c^Ty \ge \alpha c^Tw\ge \alpha$ whenever $y = \alpha w$ for some $\alpha \ge 0$ and $w\in \Cscr'$,
    from which the inequality follows.

    We now prove \eqref{sec4:eq}.
    Suppose that $y\in {\rm cl}({\rm dom}(g))\backslash ({\rm dom}(g))$. It then follows from Lemma~\ref{sec4:lem1} that $\inf_{c\in \Cscr}c^Ty\ge 0$.
    If $\inf_{c\in \Cscr}c^Ty\ge \lambda > 0$, then it follows from definition that $y\in \lambda \Cscr'\subseteq {\rm cone}(\Cscr') = {\rm dom}(g)$, a contradiction.
    Hence, we must have $\inf_{c\in \Cscr}c^Ty=0$.

    Suppose now that $y\in {\rm dom}(g)$. Then $g(y)\ge 0$. If $\lambda := \inf_{c\in \Cscr}c^Ty = 0$, then equality holds.
    On the other hand, if $\lambda > 0$, then $c^T\frac{y}{\lambda}\ge 1$ for all $c\in \Cscr$, meaning that $\frac{y}{\lambda}\in \Cscr'$.
    Using properties {\rm (iii)} and {\rm (iv)} of $g$ in Lemma~\ref{sec4:lem0}, we conclude that $g(y)\ge \lambda$. Thus, $\inf_{c\in \Cscr}c^Ty= g(y)$.

    Finally, suppose in additional that $\Cscr$ is ray-like.
    We only need to consider the case when $y\notin {\rm cl}({\rm dom}(g))$ so that $g(y)=-\infty$. Suppose to the contrary that
    $\inf_{c\in \Cscr}c^Ty\ge 0$.
    Then $y\in \Cscr^*={\rm cl}({\rm cone}(\Cscr'))$, which is a contradiction. Thus, $c^Ty<0$ for some $c\in \Cscr$. Since
    the set $\Cscr$ is ray-like, we immediately have $\inf_{c\in \Cscr}c^Ty = -\infty$. Hence, equality also holds in this case.
  \end{proof}

  Now, we are ready to study gauge duality. Let $\kappa$ be a proper closed convex gauge function, i.e., the Minkowski functional of
  a closed convex set containing $0$. Recall that the polar of $\kappa$ is defined as
  \[
  \kappa^\circ(y) := \sup_{\kappa(x)\le 1}x^Ty.
  \]
  This function is a proper closed convex gauge function; see \cite[Theorem~15.1]{Roc70}.
  We denote the optimal value of \eqref{eq:1} by $v_{p}$ and that of \eqref{eq:2dual} by $v_{gd}$
  Moreover, recall that the Fenchel dual problem to \eqref{eq:1} is given by
  \begin{equation}
    \label{sec4:eq1}
    \maximize{x} \quad -\sigma_{\Cscr}(-y) \textt{subject to} \kappa^\circ(y)\le 1,
  \end{equation}
  whose optimal value is denoted by $v_{fd}$. Then we have the following result relating these values.
  \begin{theorem}\label{sec4:thm1}
  Suppose that ${\rm dom}(\kappa^\circ)\cap \Cscr'\neq \emptyset$.
  Then it holds that
  \[
  v_p\ge v_{fd} = \frac{1}{v_{gd}} > 0.
  \]
  Furthermore, if $y^*$ solves \eqref{sec4:eq1}, then $y^*\in {\rm cone}(\Cscr')$
  and $\frac{1}{g(y^*)}y^*$ solves \eqref{eq2:dual}. Finally, if $w^*$ solves \eqref{eq2:dual}, then $\frac{1}{v_{gd}} w^*$ solves \eqref{sec4:eq1}.
%    Consider the problems
%    \begin{equation}\label{primal}
%      \begin{array}{rl}
%        v^* := \displaystyle\min_x & f(x)\\
%               {\rm s.t.} & x\in C,
%      \end{array}
%    \end{equation}
%    and
%    \begin{equation}\label{dual}
%      \begin{array}{rl}
%        w^* := \displaystyle\min_y & f^\circ(y)\\
%               {\rm s.t.} & y\in C'.
%      \end{array}
%    \end{equation}
%    . Then $v^*\ge {w^*}^{-1}$ and $w^*$ is finite (could be zero).
%    If in addition we have ${\rm ri}({\rm dom}(f))\cap {\rm ri}(C)\neq \emptyset$ and $C$ is ray-like, then $v^* = {w^*}^{-1}$
%    and $w^*$ is attained.
  \end{theorem}
  \begin{proof}
    The fact that $v_p\ge v_{fd}$ follows from standard Fenchel duality theory. We now show that $v_{fd} = \frac{1}{v_{gd}}$.

    To this end, note that
    \begin{equation}\label{sec4:ineq2}
    -\sigma_\Cscr(-y) = -\sup_{c\in \Cscr} c^T(-y) = \inf_{c\in \Cscr}c^Ty.
    \end{equation}
    Since ${\rm dom}(\kappa^\circ)\cap \Cscr'\neq \emptyset$, there exists $y_0$ so that $\kappa^\circ(y_0)\le 1$
    and $y_0\in t\Cscr'$ for some $t>0$. Hence, in particular, we have
    \begin{equation}\label{sec4:eq3}
    v_{fd}\ge \inf_{c\in \Cscr}c^Ty_0 \ge t > 0.
    \end{equation}
    Moreover, using \eqref{sec4:ineq2} and the definition of $v_{fd}$, we have
    \begin{equation*}
      \begin{split}
        v_{fd} &= \sup\{-\sigma_{\Cscr}(-y):\; \kappa^\circ(y)\le 1\} = \sup\{g(y):\; \kappa^\circ(y)\le 1\},
      \end{split}
    \end{equation*}
    where the equality holds because of Lemma~\ref{sec4:lem2}, the observation that $-\sup_{c\in \Cscr} c^T(-y) = \inf_{c\in \Cscr}c^Ty$ is negative for any $y\notin {\rm cl}({\rm dom}(g)) = {\rm cl}({\rm cone}(\Cscr')) = \Cscr^*$, and the positivity of $v_{fd}$ from \eqref{sec4:eq3}.

    Then, we have further that
    \begin{equation}\label{sec4:vineq}
      \begin{split}
        v_{fd}
        & = \sup_y\{g(y):\; \kappa^\circ(y)\le 1, y \in {\rm cone}(\Cscr')\}\\
        & = \sup_y\{g(y):\; \kappa^\circ(y)\le 1, y \in {\rm cone}(\Cscr')\backslash\{0\}\},\\
      \end{split}
    \end{equation}
    where the first equality follows from Lemma~\ref{sec4:lem0}{\rm (i)}, and the second equality follows from
    the assumption that ${\rm dom}(\kappa^\circ)\cap \Cscr'\neq \emptyset$.

    Now, observe that
    \begin{equation}\label{sec4:rel1}
    \begin{split}
     &\sup_y\{g(y):\; \kappa^\circ(y)\le 1, y \in {\rm cone}(\Cscr')\backslash\{0\}\}\\
     = & \sup_y\{g(y):\; \kappa^\circ\left(\frac{y}{g(y)}\right)\le \frac{1}{g(y)}, y \in {\rm cone}(\Cscr')\backslash\{0\}\}\\
     \le & \sup_{\lambda,u} \left\{\lambda:\; \kappa^\circ(u)\le \frac{1}{\lambda} ,\lambda > 0, u\in \Cscr'\right\},
    \end{split}
    \end{equation}
    where the equality follows from Lemma~\ref{sec4:lem0}{\rm (iii)} so that $g(y) > 0$ on the feasible set, and
    the inequality follows from Lemma~\ref{sec4:lem0}{\rm (v)}. On the other hand, we also have
    \begin{equation}\label{sec4:rel2}
    \begin{split}
    \sup_{\lambda,u} \left\{\lambda:\; \kappa^\circ(u)\le \frac{1}{\lambda} ,\lambda > 0, u\in \Cscr'\right\}
    &\le  \sup_{\lambda,u} \left\{\lambda:\; \kappa^\circ(u)\le \frac{g(u)}{\lambda} ,\frac{\lambda}{g(u)} > 0, u\in \Cscr'\right\}\\
    &= \sup_{\alpha,u}\left\{\alpha g(u):\; \alpha \kappa^\circ(u)\le 1, \alpha > 0, u\in \Cscr'\right\}\\
    &= \sup_y\{g(y):\; \kappa^\circ(y)\le 1, y \in {\rm cone}(\Cscr')\backslash\{0\}\},
    \end{split}
    \end{equation}
    where the inequality follows from Lemma~\ref{sec4:lem0}{\rm (iii)}, and the first equality follows from
    the substitution $\alpha = \frac{\lambda}{g(u)}$. In particular, equality holds throughout \eqref{sec4:rel1} and \eqref{sec4:rel2}.
    Combining these with \eqref{sec4:vineq}, we conclude that
    \[
    v_{fd} = \sup_{\lambda,u} \left\{\lambda:\; \kappa^\circ(u)\le \frac{1}{\lambda} ,\lambda > 0, u\in \Cscr'\right\} = \frac{1}{v_{gd}}.
    \]

    Finally, assume that $y^*$ solves \eqref{sec4:eq1}. Then we see immediately from \eqref{sec4:rel1} that $y^*\in {\rm cone}(C')$
    and $\frac{1}{g(y^*)}y^*$ solves \eqref{eq2:dual}. On the other hand, if $w^*$ solves \eqref{eq2:dual}, one can observe from \eqref{sec4:rel2}
    that $\frac{1}{v_{gd}} w^*$ solves \eqref{sec4:eq1}. This completes the proof.


%    Finally, assume that the constraint qualification ${\rm ri}({\rm dom}(f))\cap {\rm ri}(C)\neq \emptyset$ holds and
%    that $C$ is ray-like. The constraint qualification asserts that equality holds in \eqref{ineq} and also $v^*$ is finite.
%    Since $v^*\ge {w^*}^{-1}$, it must hold that $v^* > 0$.
%    On the other hand, since $C$ is ray-like, we see from Lemma~\ref{sec4:lem2} that the domain of the function $\inf_{c\in C}c^Ty$ is contained
%    in ${\rm cl}({\rm dom}(g))$. Hence, we have
%    \[
%    \begin{split}
%      0<v^* = \sup_y\left\{\inf_{c\in C}c^Ty:\; f^\circ(y)\le 1\right\}
%      &= \sup_y\left\{\inf_{c\in C}c^Ty:\; f^\circ(y)\le 1, y\in {\rm cl}({\rm dom}(g)) \right\}\\
%      &= \sup_y\left\{\inf_{c\in C}c^Ty:\; f^\circ(y)\le 1, y\in {\rm cl}({\rm cone}(C')) \right\}\\
%      &= \sup_y\left\{\inf_{c\in C}c^Ty:\; f^\circ(y)\le 1, y\in {\rm cone}(C') \right\}\\
%      &= \sup_y\left\{g(y):\; f^\circ(y)\le 1, y\in {\rm cone}(C') \right\},
%    \end{split}
%    \]
%    where the third equality follows from Lemma~\ref{sec4:lem2}, which states
%    that $\inf_{c\in C}c^Ty = 0$ for any $y\in {\rm cl}({\rm dom}(g))\backslash ({\rm dom}(g))$.
%    Hence, equality also holds throughout \eqref{vineq}. The attainment of $w^*$
%    is just the dual attainment result in Fenchel duality.
  \end{proof}

  From the well studied Fenchel duality theory, we have the following immediate corollary. {\color{red} Can write down more conditions, e.g. polyhedral, blablabla...}
  \begin{corollary}
    Suppose that ${\rm dom}(\kappa^\circ)\cap \Cscr'\neq \emptyset$ and ${\rm ri}({\rm dom}(\kappa))\cap {\rm ri}(\Cscr)\neq \emptyset$.
    Then $v_pv_{gd}=1$ and the gauge dual problem \eqref{eq:2dual} is attained.
  \end{corollary}
  \begin{proof}
    From ${\rm ri}({\rm dom}(\kappa))\cap {\rm ri}(\Cscr)\neq \emptyset$ and \cite[Theorem~31.1]{Roc70}, we see that $v_p = v_{fd}$ and $v_{fd}$
    is attained. The conclusion of the corollary now follows immediately from Theorem~\ref{sec4:thm1}.
  \end{proof}

  We would also like to guarantee {\em primal attainment}. Notice that the gauge dual of the gauge dual problem \eqref{eq:2dual} is given by
  \begin{equation}
    \label{sec4:eq4}
    \minimize{x} \quad \kappa(x) \textt{subject to} x\in\Cscr'',
  \end{equation}
  which is not the same as \eqref{eq:1} unless $\Cscr$ is ray-like.
  However, we show in the next proposition that \eqref{sec4:eq4} and \eqref{eq:1} always have
  the same optimal value, and that if the optimal value is attained in one problem, so is in the other one.

  \begin{proposition}\label{sec4:prop1}
    The optimal value of \eqref{eq:1} and \eqref{sec4:eq4} are the same. Moreover, if
    the optimal value is attained in one problem, so is in the other one.
  \end{proposition}
  \begin{proof}
    It is not hard to show that $\Cscr'' = \bigcup_{\lambda \ge 1}\lambda \Cscr$ for a closed convex set $\Cscr$ that does not contain the origin;
    see also \cite[Page~176]{McLinden:1978}. Thus,
    \eqref{sec4:eq4} with $\Cscr''$ in place of $\Cscr$ is equivalent to
    \begin{equation}\label{sec4:primal2}
      \minimize{\lambda,x} \quad \lambda\kappa(x) \textt{subject to} x\in\Cscr,\ \lambda\ge 1,
    \end{equation}
    which clearly gives the same optimal value as \eqref{eq:1}. This proves the first conclusion.
    The second conclusion now also follows immediately.
  \end{proof}

  Hence, we obtain the following corollary, which generalizes \cite[Theorem~2A]{freund:1987} by dropping the ray-likeness assumption
  on $\Cscr$.
  \begin{corollary}
    Suppose that ${\rm ri}({\rm dom}\kappa)\cap {\rm ri}(\Cscr)\neq \emptyset$ and ${\rm ri}({\rm dom}\kappa^\circ)\cap {\rm ri}(\Cscr')\neq \emptyset$.
    Then $v_pv_{gd}=1$ and both values are attained.
  \end{corollary}
  \begin{proof}
    The conclusion follows from Theorem~\ref{sec4:thm1}, Proposition~\ref{sec4:prop1}
    and the observation that ${\rm ri}({\rm dom}(\kappa))\cap {\rm ri}(\Cscr)\neq \emptyset$
    if and only if ${\rm ri}({\rm dom}(\kappa))\cap {\rm ri}(\Cscr'')\neq \emptyset$, since
    ${\rm ri}(\Cscr'') = \bigcup_{\lambda > 1}\lambda {\rm ri}(\Cscr)$ (see \cite[Page~50]{Roc70})
    and ${\rm dom}(\kappa)$ is a cone.
  \end{proof}














  \subsection{Examples}
  \begin{itemize}
  \item nonnegative CP requirements for strong duality: what
    conditions on the NNCP satisfy the strong-duality conditions
    above.
    \begin{itemize}
    \item Ives assumed $c \in \int(\Kscr)$. Relax?
    \item contrast these conditions to those required for conic
      programming for Lagrange duality
    \end{itemize}
  \item questions:
    \begin{itemize}
    \item example where strong GD holds, but strong LD does not
    \item what is the difference between strengths of GD and LD
      gaps? (See Lemma 3 of Freund)
    \end{itemize}
  \end{itemize}

  \section{Sensitivity analysis}
  \begin{itemize}
  \item concept of gauge multipliers?
  \item subdifferential of the gauge value function?
  \item perturbations in $b$
  \end{itemize}

  \section{Algorithms}
  \label{sec:algorithms}

  \begin{itemize}
  \item A generic algorithm for gauge duals?
  \end{itemize}




  \section{Conclusion}


  \bibliographystyle{plainnat}
  \bibliography{test,master}

\end{document}
